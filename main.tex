\documentclass{article}
\usepackage{graphicx} % Required for inserting images

\title{CSE550-project}
\author{CSE550 Project}
\date{November 2025}

\begin{document}

\maketitle

\section{Introduction}

This project focuses on ``using machine learning for quantum state estimation.'' Classical estimators covered in the course---ranging from maximum likelihood tomography to Kalman filtering---provide the baselines against which learning-based approaches are evaluated. The following review captures the first phase of the project, highlighting the state of the art, domain boundaries, and the complementary relationships between traditional and learning-centric methods.

\section{Landscape Review: Machine-Learning-Based Quantum State Estimation}

The goal of quantum state estimation is to reconstruct a density matrix or wave function as accurately as possible under limited measurement budgets. Projection-based tomography and statistical filters such as the Kalman family perform well for low-dimensional systems with well-controlled noise. However, the rise of noisy intermediate-scale quantum (NISQ) devices and large many-body systems exposes key limitations in sample complexity, computational cost, and adaptability, creating opportunities for machine learning.

\subsection{Classical Quantum State Estimation}

Maximum-likelihood quantum tomography repeatedly measures and iteratively refines state estimates, offering robust performance for small laboratory systems but suffering from overfitting and numerical instability as the parameter dimension grows exponentially\cite{Hradil1997}. When the underlying state is sparse or low rank, compressed sensing reduces the measurement complexity from exponential to polynomial order, which is particularly valuable for calibrating superconducting qubit arrays\cite{Gross2010}. For continuously monitored systems, quantum Kalman filtering models signal and noise dynamics jointly, enabling incremental estimation with real-time feedback and forming the backbone of many quantum optics platforms\cite{Wiseman2010}. These methods rely heavily on explicit physics priors and often struggle when measurement noise, device drifts, or unmodeled correlations dominate.

\subsection{Machine-Learning-Driven Estimators}

Machine learning aims to alleviate dependence on explicit models by leveraging flexible data-driven representations. Neural-network quantum state tomography (NNQST) parameterizes amplitudes and phases via restricted Boltzmann machines or neural autoregressive flows, reconstructing highly entangled states with fewer samples\cite{Torlai2018}. Generative models such as Born machines and variational autoencoders replicate measurement distributions while providing differentiable latent representations for downstream tasks like error mitigation or phase estimation\cite{Carrasquilla2019}. Recent work combines meta-learning and classical shadows to transfer estimators across Hamiltonians and noise configurations, predicting a broad set of observables with near-polynomial numbers of measurements\cite{Huang2020}. Compared with maximum likelihood and Kalman filters, these learning-based strategies offer greater flexibility for nonlinear modeling and many-body scaling, at the cost of increased data requirements, limited interpretability, and the need to enforce physical consistency constraints.

\subsection{Open Challenges and Course Connections}

Key open problems include: (i) designing loss functions that remain differentiable yet respect positivity and trace constraints; (ii) training high-dimensional models with limited experimental data while supporting online or adaptive updates; and (iii) benchmarking classical and learning-based estimators under multi-body noise, parameter drift, and hardware nonidealities. Subsequent project phases will revisit the classical techniques presented in class to establish principled comparisons and guide experimental designs for these open questions.

\begin{thebibliography}{9}
\bibitem{Hradil1997}
Z.~Hradil, ``Quantum-state estimation,'' \textit{Phys. Rev. A}, vol.~55, pp.~R1561--R1564, 1997.

\bibitem{Gross2010}
D.~Gross, Y.-K.~Liu, S.~T.~Flammia, S.~Becker, and J.~Eisert, ``Quantum state tomography via compressed sensing,'' \textit{Phys. Rev. Lett.}, vol.~105, p.~150401, 2010.

\bibitem{Wiseman2010}
H.~M.~Wiseman and G.~J.~Milburn, \textit{Quantum Measurement and Control}. Cambridge University Press, 2010.

\bibitem{Torlai2018}
G.~Torlai, G.~Mazzola, J.~Carrasquilla, et al., ``Neural-network quantum state tomography,'' \textit{Nat. Phys.}, vol.~14, pp.~447--450, 2018.

\bibitem{Carrasquilla2019}
J.~Carrasquilla, G.~Torlai, R.~G.~Melko, and L.~Aolita, ``Reconstructing quantum states with generative models,'' \textit{Nat. Mach. Intell.}, vol.~1, pp.~155--161, 2019.

\bibitem{Huang2020}
H.-Y.~Huang, R.~Kueng, and J.~Preskill, ``Predicting many properties of a quantum system from very few measurements,'' \textit{Nat. Phys.}, vol.~16, pp.~1050--1057, 2020.
\end{thebibliography}

\end{document}
